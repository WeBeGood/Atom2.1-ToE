\documentclass[11pt]{article}
\usepackage{amsmath,amssymb}
\usepackage{geometry}
\geometry{margin=1in}

\title{N015: Trinity Phase-Lock Seed Event (Two-Wave $\to$ Newmode)}
\author{WeBeGood (Atom 2.1 ToE)}
\date{}

\begin{document}
\maketitle

\section{Setup}
We consider two EM excitations with fields $(\mathbf{E}_1,\mathbf{B}_1)$ and $(\mathbf{E}_2,\mathbf{B}_2)$.
A first-pass model uses linear Maxwell superposition:
\begin{align}
\mathbf{E} &= \mathbf{E}_1 + \mathbf{E}_2,\\
\mathbf{B} &= \mathbf{B}_1 + \mathbf{B}_2.
\end{align}
A relative phase parameter $\theta$ controls the alignment of the oscillatory components.

\section{Trinity phase-lock hypothesis}
Hypothesis: there exists a stability extremum near $\theta \approx 2\pi/3$ (120$^\circ$) for a gauge-invariant turning-region functional $\mathcal{S}(\mathbf{E},\mathbf{B})$.

\section{Gauge invariance}
Any seed criterion must be expressible in terms of gauge-invariant quantities (e.g., $\mathbf{E},\mathbf{B}$ or $F,\star F$).

\bibliographystyle{plain}
\bibliography{refs,../../refs/global}
\end{document}
