\documentclass[11pt]{article}
\usepackage{amsmath,amssymb}
\usepackage{geometry}
\geometry{margin=1in}
\title{N000: Space, Time, Light (Maxwell Baseline)}
\author{WeBeGood (Atom 2.1 ToE)}
\date{}
\begin{document}
\maketitle

\section{Baseline Assumptions}
Vacuum: $\rho=0$, $\mathbf{J}=0$. Cartesian coordinates $(x,y,z)$ and time $t$.

\section{Maxwell Equations (Vacuum)}
\begin{align}
\nabla\cdot\mathbf{E} &= 0\\
\nabla\cdot\mathbf{B} &= 0\\
\nabla\times\mathbf{E} &= -\frac{\partial\mathbf{B}}{\partial t}\\
\nabla\times\mathbf{B} &= \mu_0\varepsilon_0\frac{\partial\mathbf{E}}{\partial t}
\end{align}

\section{Wave Equation}
Take $\nabla\times$ of Faraday and substitute Amp\`ere--Maxwell to obtain:
\begin{equation}
\nabla^2\mathbf{E} = \mu_0\varepsilon_0\frac{\partial^2\mathbf{E}}{\partial t^2},\qquad
\nabla^2\mathbf{B} = \mu_0\varepsilon_0\frac{\partial^2\mathbf{B}}{\partial t^2}.
\end{equation}
Thus $c = 1/\sqrt{\mu_0\varepsilon_0}$.

\bibliographystyle{plain}
\bibliography{refs}
\end{document}
